%% LaTeX2e class for student theses
%% sections/content.tex
%% 
%% Karlsruhe Institute of Technology
%% Institute for Program Structures and Data Organization
%% Chair for Software Design and Quality (SDQ)
%%
%% Dr.-Ing. Erik Burger
%% burger@kit.edu
%%
%% Version 1.3.2, 2017-08-01

\chapter{Motivation}
\label{ch:Motivation}

\begin{itemize}
	\item show Motivation of research
	\item show current practice how done currently
	\item outlook: desired state 
\end{itemize}


\section{Worldwide LHC Computing Grid }
\label{sec:Motivation:wlcg}
\begin{itemize}
	\item explain what wlcg is
	\item which experiment? CMS
	\item some specifications: jobs, cpus, ...
	\item highly heterogeneous, lot of constraints, like bad connection of some datacenters
	\item \cite{bayatian2005cms} \cite{bonacorsi2007cms} \cite{hernandez2008cms}
\end{itemize}

\section{State of Practice}
\label{sec:Motivation:State of Practice}

\begin{itemize}
	\item load balancing manually
	\item operators look at monitoring data and send new jobs to  compute node (datacenter) which they think is best
	\item heavily influenced by experience
	\item not optimal, shown by motoring data
	\item does not account nature of jobs (io vs compute), leading to not good utilization of the nodes
	\item desired: full utilization of nodes by submitting the right amount of io and compute job, so that hdd and cpu are utilized
	\item example too much io jobs: cpu are idling, wasting time
\end{itemize}


\section{Outlook}
\label{sec:Motivation:Outlook}

\begin{itemize}
	\item simulate effect of different load balancing strategies and decide on these results what the best scheduling is
	\item for that first a model of the wlcg is needed
	\item allows on the one hand to optimize scheduling
	\item on the other hand evaluate what happens if grid is changed, like dynamically adding Amazon nodes when price is low
\end{itemize}



\chapter{State of the Art}
\label{ch:SecondContent}

\begin{itemize}
	\item selection of research which related 
	\item show where they lack and own is needed
\end{itemize}

\section{Resource management for Infrastructure as a Service (IaaS) in cloud computing: A survey}
\label{sec:StateOfTheArt:SurveyResouceManagement}
\begin{itemize}
	\item \cite{manvi2014resource}
	\item overview about problems in IaaS
\end{itemize}

\section{Cloud Simulators}

\subsection{Rapid Testing of IaaS Resource Management Algorithms via Cloud Middleware Simulation}
\begin{itemize}
	\item How to test load balancing alorithm. Able to test algorithm directly without reimplementing them for specific simulator.
	\item CACTOS Runtime Toolkit integrates monitoring and resource management via a variety of algorithms.
	\item The CACTOS Prediction Toolkit: Cloud simulator with the ability to evaluate
	resource management algorithms without modification
	\item cloud simulators: CloudSim, GreenCloud
	\item missing: several date centers, which are connected, local data etc.
\end{itemize}

\subsection{A survey of mathematical models, simulation approaches and testbeds used for research in cloud computing}
\label{sec:StateOfTheArt:SurveyClouds}
\begin{itemize}
	\item \cite{survey_clouds}
	\item lot of simulations
\end{itemize}

\subsection{CloudSim}
\begin{itemize}
	\item \cite{cloud_sim}
	\item mostly used 
	\item lot of simulators are based on it
\end{itemize}

\subsection{CDOSim}
\begin{itemize}
	\item \cite{cdosim}
	\item based on cloudsim
	\item represent more user than provider perspective
\end{itemize}

\subsection{Emusim}
\begin{itemize}
	\item \cite{emusim}
	\item Profiling based Approach to extract Workload Models
	\item simulates behaviour of application
\end{itemize}

\subsection{Cloud Simulator for Autoscaling}
\begin{itemize}
	\item \cite{autoscale_cloud}
	\item based on queueing models
	\item allows to evaluate autoscaling algorithms
\end{itemize}

\subsection{Locality Sim: Cloud Simulator with Data Locality}
\begin{itemize}
	\item \cite{localitysim}
	\item based on cloudsim
	\item considers data-locality
\end{itemize}

\subsection{NetworkCloudSim}
\begin{itemize}
	\item \cite{networkcloudsim}
	\item extends CloudSim
	\item models network
\end{itemize}


\subsection{GreenCloud}
\begin{itemize}
	\item \cite{green_cloud}
	\item energy-aware of severs, switches and links
	\item energy efficiency
	\item packet level
\end{itemize}

\subsection{MDCSim}
\begin{itemize}
	\item \cite{mdcsim}
	\item multi-tier data centers
	\item detailed implementation of each tier
\end{itemize}

\subsection{Palladio}
\begin{itemize}
	\item architectural templates \cite{arch}
	\item black box resource demand \cite{blackbox}
\end{itemize}


\subsection{CACTOS}
\begin{itemize}
	\item \cite{cactos}
	\item CACTOS Runtime Toolkit: monitoring and resource management
	\item install cactos on servers to monitor and manage them
	\item CACTOS Prediction Toolkit: evaluation of alternative data center
	deployment scenarios, and resource management algorithms
	\item uses PCM and SimuLizar
\end{itemize}

\section{Grid Simulators}
\subsection{GridSim}
\begin{itemize}
	\item \cite{gridsim}
	\item foundation for cloudsim
	\item best developed
\end{itemize}

\subsection{OptorSim}
\begin{itemize}
	\item \cite{optorsim}
	\item used to evaluate data replication strategies
\end{itemize}

\subsection{SimGrid}
\begin{itemize}
	\item \cite{simgrid}
	\item framework for simulation of distributed applications in Grid platforms
\end{itemize}

\subsection{DGSim}
\begin{itemize}
	\item \cite{dgsim}
	\item trace based
	\item automatizes the simulation process
	\item generating realistic grid systems and workloads
\end{itemize}

\subsection{ChicagoSim}
\begin{itemize}
	\item \cite{chicagosim}
	\item Data Grids
	\item respects data locality
\end{itemize}


\subsection{Differences Grid and Cloud}
\begin{itemize}
	\item \cite{compare_grid_cloud}
	\item cloud virtualized resources
	\item our case rather grid
\end{itemize}


